\documentclass[12pt]{article}                                                                                                                       
\usepackage{sbc-template}                                                 
\usepackage{graphicx,url}                                                 
\usepackage[utf8]{inputenc}                                               
\usepackage[brazil]{babel}                                                      

\title{Religiões do Mundo\\O Islã}
\author{Iyan Lucas Duarte Marques\inst{1}
Matheus Costa Faria\inst{1}
Camila Moreira Lopes\inst{1}}

\address{Instituto de Ciências Exatas e Informática - Pontifícea Universidade Católica Minas Gerais (PUC-MG)}

\begin{document}

\maketitle
\section{Introdução}
O Islã é uma religião do tronco judaico originária na península arábica com a suas figuras centrais, o profetá maior Muhhamed e o único deus do panteão, Alá.
Atualmente, a religião é segmentada por três grandes escolas, a Sunita, a Xiita e a Ibadi. 
Cada escola possui também as suas ramificações com interpretações próprias do livro sagrado Alcorão, adquirindo um caráter descentralizado, ou seja, sem uma figura líder central na religião.
Desta forma, o Islã não possui uma instituição forte como o ramo cristão.
Os muçulmanos acreditam que Alá é único e incomparável e o propósito da existência é adorá-lo.
Eles também acreditam que o Islã é a versão completa e universal de uma fé primordial que foi revelada em muitas épocas e lugares anteriores, incluindo por meio de Abraão, Moisés e Jesus, que eles consideram profetas.
Os seguidores do Islã afirmam que as mensagens e revelações anteriores foram parcialmente alteradas ou corrompidas ao longo do tempo, mas consideram o Alcorão como uma versão inalterada da revelação final de Alá.
A maioria dos muçulmanos vivem majoritariamente na África e sul da Ásia, des do Magarebe às ilhas malacas.
\section{Origem/matriz cultural}

\subsection{Fundador}
\subsection{Livros sagrados}
\section{Aspectos Simbólicos}
\subsection{Rituais}
\subsection{Papel masculino e Feminino}
\subsection{Alá}
\subsection{Mitos originários}
\subsection{Orientações para a Vida e a Morte}
\subsection{Valores}
\section{O Islã no Brasil}
\subsection{Localização}
\subsection{número de adeptos Brasil}





\end{document}